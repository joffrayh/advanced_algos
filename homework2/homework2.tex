\documentclass[11pt, english]{article}

\title{\textbf{COM3105 Advanced Algorithms} \\ Homework 2}
\author{Joffray Hargreaves}
\date{\vspace{-3em}} % Moves the date up or removes it completely

\usepackage[a4paper,
            bindingoffset=0in,
            left=1in,
            right=1in,
            top=1in,
            bottom=1in,
            footskip=.25in]{geometry}

\usepackage{amsthm} % For theorem environments
\usepackage{amsmath} % For advanced math typesetting
\usepackage{amssymb} % For \emptyset
\usepackage{bm} % For bold math symbols
\usepackage{setspace} % For line spacing
\setstretch{1.1} % Slightly increased line spacing for readability

\renewcommand{\labelitemi}{$\bullet$}
\renewcommand{\labelitemii}{$\circ$}
\renewcommand{\labelitemiii}{$\diamond$}

% --- Custom Environments ---

% Question environment: with bottom rule and marks aligned right
\newenvironment{question}[1][]{%
  \par\noindent{\Large\bfseries Question}%
  \ifx#1\empty\else\hfill{\normalsize\itshape(#1)}\fi%
  \par\vspace{0.25\baselineskip}\noindent\ignorespaces
}{%
  \par\vspace{1\baselineskip}%
}

% Answer environment: with top and bottom rule, no gap above heading
\newenvironment{answer}{%
  \par\noindent\hrule % top line
  \vspace{0.25\baselineskip}%
  \noindent{\Large\bfseries Answer}\par\vspace{0.25\baselineskip}\noindent\ignorespaces
}{%
  \par\vspace{0.5\baselineskip}%
  \noindent\hrule\vspace{0.5\baselineskip}%
  \hrule\hrule
}


% --- Document Start ---
\begin{document}
\maketitle

\begin{question}[5 points]
Given a simple graph $G = (V, E)$ (no loops or multiple edges), 
consider the following deterministic algorithm: 
\begin{itemize}
\item start with an arbitrary partition. 
\item If moving a vertex from one part to the other increases the number of crossing edges, we will move it. 
\item We do this until there is no such vertex left.
\end{itemize}

\noindent Why does this algorithm terminate? \\
What can you say about the number of edges crossing when the algorithm terminates?
\end{question}

\begin{answer}
\noindent \textbf{Termination of the Algorithm:}
\begin{itemize}
    \item Let $C$ be the set of edges crossing the partition. 
    \item Let state $i$ be the state of the partition after $i$ moves have been made.
    \item For each state $i$, we iteratively check a vertex $v_{j}$ (where  $0 \leq j < |V|$) to see if moving it increases $|C|$. 
    \item If moving a vertex $v_{j}$ increases $|C|$ we make this move and move to state $i+1$ and restart the same process. 
    \item Checking each state $i$ takes $\leq |V|$ steps to check, where $V$ is finite. 
    \item As $V$ is finite, for a simple graph, $E$ is aslo finite, so the maximum $|C|$ is $|E|$. 
    \item Therefore, in the worst case:
    \begin{itemize}
        \item at state 0,
          \begin{itemize}
              \item $|C|$ = 0 (no edges in partition)
              \item Adding $v_{|V|-1}$, increases $|C|$ by 1.
          \end{itemize}
        \item at state 1,
          \begin{itemize}
              \item $|C|$ = 1
              \item Adding $v_{|V|-1}$, increases $|C|$ by 1.
          \end{itemize}
        \item at state i,
          \begin{itemize}
                \item $|C|$ = i
                \item Adding $v_{|V|-1}$, increases $|C|$ by 1.
          \end{itemize}
        \item at state $|E|$,
          \begin{itemize}
              \item $|C|$ = $|E|$
              \item Adding $v_{|V|-1}$, cannot increase $|C|$ by 1, as $|C|$ is saturated.
              \item Algorithm stops as no more edges to add.
          \end{itemize}
        \item So $|E|$ edges being added to $C$ and taking $|V|$ checks at each state.
        \item Hence the runtime is $\mathcal{O}(|E| * |V|)$.
        \item Both $E$ and $V$ are finite, so the algorithm terminates
    \end{itemize}
\end{itemize}

\noindent\textbf{Number of Crossing Edges at Termination:} \\
As the algorithm doesn't make any clever choices about which vertex to move, it is possible that the algorithm terminates in a local maximum. \\
However, we can prove that the algorithm terminates with a minimum number of crossing edges. \\
\end{answer}

\end{document}
