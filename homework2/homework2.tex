\documentclass[11pt, english]{article}

\title{\textbf{COM3105 Advanced Algorithms} \\ Homework 2}
\author{Joffray Hargreaves}
\date{\vspace{-3em}} % Moves the date up or removes it completely

\usepackage[a4paper,
            bindingoffset=0in,
            left=1in,
            right=1in,
            top=1in,
            bottom=1in,
            footskip=.25in]{geometry}

\usepackage{amsthm} % For theorem environments
\usepackage{amsmath} % For advanced math typesetting
\usepackage{amssymb} % For \emptyset
\usepackage{bm} % For bold math symbols
\usepackage{setspace} % For line spacing
\setstretch{1.1} % Slightly increased line spacing for readability

\renewcommand{\labelitemi}{$\bullet$}
\renewcommand{\labelitemii}{$\circ$}
\renewcommand{\labelitemiii}{$\diamond$}

% --- Custom Environments ---

% Question environment: with bottom rule and marks aligned right
\newcounter{questionnum}
\newenvironment{question}[1][]{%
  \stepcounter{questionnum}%
  \par\noindent{\Large\bfseries Question \thequestionnum}%
  \ifx#1\empty\else\hfill{\normalsize\itshape(#1)}\fi%
  \par\vspace{0.25\baselineskip}\noindent\ignorespaces
}{%
  \par\vspace{1\baselineskip}%
}

% Answer environment: with top and bottom rule, no gap above heading
\newenvironment{answer}{%
  \par\noindent\hrule % top line
  \vspace{0.25\baselineskip}%
  \noindent{\Large\bfseries Answer}\par\vspace{0.25\baselineskip}\noindent\ignorespaces
}{%
  \par\vspace{0.5\baselineskip}%
  \noindent\hrule\vspace{0.5\baselineskip}%
  \hrule\hrule
}


% --- Document Start ---
\begin{document}
\maketitle

\begin{question}[5 points]
Given a simple graph $G = (V, E)$ (no loops or multiple edges), 
consider the following deterministic algorithm: 
\begin{itemize}
\item start with an arbitrary partition. 
\item If moving a vertex from one part to the other increases the number of crossing edges, we will move it. 
\item We do this until there is no such vertex left.
\end{itemize}

\noindent Why does this algorithm terminate? \\
What can you say about the number of edges crossing when the algorithm terminates?
\end{question}

\begin{answer}
\noindent \textbf{Termination of the Algorithm:}
\begin{itemize}
    \item Let $C$ be the set of edges crossing the partition. 
    \item Let state $i$ be the state of the partition after $i$ moves have been made.
    \item For each state $i$, we iteratively check a vertex $v_{j}$ (where  $0 \leq j < |V|$) to see if moving it increases $|C|$. 
    \item If moving a vertex $v_{j}$ increases $|C|$ we make this move and move to state $i+1$ and restart the same process. 
    \item Checking each state $i$ takes $\mathcal{O}(|V|)$ steps, where $V$ is finite. 
    \item As $V$ is finite, for a simple graph, $E$ is also finite, so the maximum $|C|$ is $|E|$. 
    \item Therefore, in the worst case:
    \begin{itemize}
        \item at state 0,
          \begin{itemize}
              \item $|C|$ = 0 (no edges in partition)
              \item Adding $v_{|V|-1}$, increases $|C|$ by 1.
          \end{itemize}
        \item at state 1,
          \begin{itemize}
              \item $|C|$ = 1
              \item Adding $v_{|V|-1}$, increases $|C|$ by 1.
          \end{itemize}
        \item at state i,
          \begin{itemize}
                \item $|C|$ = i
                \item Adding $v_{|V|-1}$, increases $|C|$ by 1.
          \end{itemize}
        \item at state $|E|$,
          \begin{itemize}
              \item $|C|$ = $|E|$
              \item No movement of vertex can increase $|C|$, as $|C|$ = $|E|$.
              \item Algorithm stops as no more edges to add.
          \end{itemize}
        \item So $|E|$ edges being added to $C$ and taking $|V|$ checks at each state.
        \item Hence the runtime is $\mathcal{O}(|E| * |V|)$.
        \item Both $E$ and $V$ are finite sets, so the algorithm terminates
    \end{itemize}
\end{itemize}

\noindent\textbf{Number of Crossing Edges at Termination.} \\[0.5em]
Let the final partition of $G$ be $(A, B)$.  
For each vertex $v \in V$:
\[
d_A(v) = |\{\, u \in A : (u,v) \in E \,\}|, \quad
d_B(v) = |\{\, u \in B : (u,v) \in E \,\}|.
\]

\noindent At termination, moving any vertex to the other part does not increase the number of crossing edges.  
Hence:
\[
d_B(v) \ge d_A(v) \quad \text{for all } v \in A, 
\qquad \text{and} \qquad
d_A(v) \ge d_B(v) \quad \text{for all } v \in B.
\]
This means that for every vertex $v \in V$, at least half of its incident edges cross the partition:
\[
\ d_C(v) \ge \tfrac{1}{2} d(v),
\]

where,\\
\indent$d(v) = d_A(v) + d_B(v)$ (i.e. the total degree of $v$),\\
\indent$d_C(v) = \textit{number of degrees from $v$ crossing the partition}$.\\

\medskip
\noindent Summing over all vertices gives:
\[
\sum_{v \in V} d_C(v) \ge \frac{1}{2} \sum_{v \in V} d(v)
\]
Since each crossing edge contributes to $d_C(v)$ for exactly two vertices,
\[
\sum_{v \in V} d_C(v) = 2|C|
\]
Combining these two expressions:
\[
2|C| \ge \frac{1}{2} \sum_{v \in V} d(v) = \frac{1}{2} (2|E|) = |E|
\]
Therefore,
\[
|C| \ge \frac{|E|}{2}
\]

\medskip
\noindent
\textbf{Conclusion.}  
When the algorithm terminates, the resulting partition has at least half of all edges crossing between the two parts.  
Although this partition may be only a local maximum, it guarantees that at least $\tfrac{|E|}{2}$ edges are cut.
\end{answer}

\begin{question}[5 points]
(MAX-SAT). Given a 3-CNF $F$ where every clause has width exactly 3, 
let $OPT(F)$ be the maximum number of clauses in $F$ that can be satisfied 
simultaneously, i.e., it is the maximum integer $m$ such that there is an 
assignment to the variables that satisfies $m$ clauses. \\
Give a randomised algorithm that produces an assignment that satisfies 
at least $\frac{7}{8} OPT(F)$ clauses on expectation. \\
You need to state the algorithm and give an analysis.
\end{question}

\begin{answer}
\textbf{Algorithm:}
  \begin{enumerate}
    \item  For each variable $x_i$ in $F$, assign it a value of 1 with probability $\frac{1}{2}$, and 0 with probability $\frac{1}{2}$.
    \item Evaluate each clause in $F$. If at least one literal in the clause is satisfied by the current assignment, count the clause as satisfied.
    \item Return the current assignment and the count of satisfied clauses.
  \end{enumerate}

\noindent\textbf{Analysis:}

\noindent Let $C$ be the set of all clauses in $F$.\\
\noindent Let $c$ be any clause in $C$. \\
\noindent The probability that any clause $c$ is satisfied using the random algorithm:
\begin{itemize}
\item Each clause has exactly 3 literals.
\item The probability that a specific literal is true is $\frac{1}{2}$.
\item For the clause to be unsatisfied, all 3 literals must be false.
\item The probability that the clause is unsatisfied is $\left(\frac{1}{2}\right)^3 = \frac{1}{8}$.
\item Therefore, the probability that the clause is satisfied is $1 - \frac{1}{8} = \frac{7}{8}$.
\end{itemize}

\noindent Let $X$ be the random variable representing the number of satisfied 
clauses in $F$ under the random assignment.

\noindent Using the linearity of expectation:
\[
\mathbb{E}[X] = \sum_{c \in C} \mathbb{E}[X_{c}]
\]

\noindent Where $X_{c}$ is a random variable indicating whether clause $c$ is satisfied.\\
\noindent Expectation for a clause $c$ in $C$:
\begin{align*}
\mathbb{E}[X_{c}] &= \sum_{x \in \{0,1\}} x \cdot P(X_{c} = x) \\
\mathbb{E}[X_{c}] &= 1 \cdot P(c \text{ is satisfied}) + 0 \cdot P(c \text{ is not satisfied}) \\
\mathbb{E}[X_{c}] &= P(c \text{ is satisfied}) = \frac{7}{8} \qquad \textit{(from above)}
\end{align*}

\noindent Using this in the expectation of $X$:

\begin{align*}
\mathbb{E}[X] &= \sum_{c \in C} \mathbb{E}[X_{c}] \\
\mathbb{E}[X] &= \sum_{c \in C} \frac{7}{8} \\
\mathbb{E}[X] &= |C| \cdot \frac{7}{8} \\
\end{align*}

\noindent This means that on average, the random assignment satisfies $\frac{7}{8}$ of all clauses in $F$. \\
\noindent Let $OPT(F)$ be the maximum number of clauses that can be 
satisfied simultaneously in $F$:
\[
|C| \geq OPT(F)
\]
\noindent Therefore:
\begin{align*}
  \mathbb{E}[X] &= |C| \cdot \frac{7}{8} \geq OPT(F)\cdot \frac{7}{8}\\
  \mathbb{E}[X] &\geq \frac{7}{8} OPT(F)
\end{align*}

\medskip
\noindent
\textbf{Conclusion.}  
The algorithm provided produces an assignment that satisfies at least $\frac{7}{8} OPT(F)$ clauses on expectation.
\end{answer}
\end{document}
