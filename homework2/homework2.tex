\documentclass[11pt, english]{article}

\title{\textbf{COM3105 Advanced Algorithms} \\ Homework 2}
\author{Joffray Hargreaves}
\date{\vspace{-3em}} % Moves the date up or removes it completely

\usepackage[a4paper,
            bindingoffset=0in,
            left=1in,
            right=1in,
            top=1in,
            bottom=1in,
            footskip=.25in]{geometry}

\usepackage{amsthm} % For theorem environments
\usepackage{amsmath} % For advanced math typesetting
\usepackage{amssymb} % For \emptyset
\usepackage{bm} % For bold math symbols
\usepackage{setspace} % For line spacing
\setstretch{1.1} % Slightly increased line spacing for readability

% --- Custom Environments ---

% Question environment: with bottom rule and marks aligned right
\newenvironment{question}[1][]{%
  \par\noindent{\Large\bfseries Question}%
  \ifx#1\empty\else\hfill{\normalsize\itshape(#1)}\fi%
  \par\vspace{0.25\baselineskip}\noindent\ignorespaces
}{%
  \par\vspace{1\baselineskip}%
}

% Answer environment: with top and bottom rule, no gap above heading
\newenvironment{answer}{%
  \par\noindent\hrule % top line
  \vspace{0.25\baselineskip}%
  \noindent{\Large\bfseries Answer}\par\vspace{0.25\baselineskip}\noindent\ignorespaces
}{%
  \par\vspace{0.5\baselineskip}%
  \noindent\hrule\vspace{0.5\baselineskip}%
  \hrule\hrule
}


% --- Document Start ---
\begin{document}
\maketitle

\begin{question}[5 points]
Given a simple graph $G = (V, E)$ (no loops or multiple edges), 
consider the following deterministic algorithm: 
\begin{itemize}
\item start with an arbitrary partition. 
\item If moving a vertex from one part to the other increases the number of crossing edges, we will move it. 
\item We do this until there is no such vertex left.
\end{itemize}

Why does this algorithm terminate? \\
What can you say about the number of edges crossing when the algorithm terminates?
\end{question}

\begin{answer}
\noindent \textbf{Termination of the Algorithm:} \\
The algorithm terminates because at each step, we either find a vertex that can be moved to increase the number of crossing edges, or we reach a point where no such vertex exists. In the latter case, we have found a local optimum with respect to the crossing edges, and the algorithm stops.

\noindent \textbf{Number of Crossing Edges at Termination:} \\
When the algorithm terminates, the number of edges crossing the partition is at a local minimum. However, this does not guarantee that it is the global minimum. The final partition may still have a significant number of crossing edges, depending on the initial configuration and the structure of the graph.
\end{answer}

\end{document}
